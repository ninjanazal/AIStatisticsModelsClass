\documentclass[a4paper]{article}
\usepackage[utf8]{inputenc}
\usepackage{geometry}

\geometry{
  top=1.2in,
  bottom=1.5in,
  left=1.1in,
  right=1.1in,
  headheight=10pt,
  headsep=10pt,
  footskip=50pt
}

\usepackage{graphicx}
\usepackage{lipsum} 
\usepackage{xcolor} 
\usepackage[hidelinks]{hyperref}
\usepackage{multirow}

\usepackage{caption}
\usepackage{float}
\usepackage{mathtools}

\usepackage{titlesec}
\titleformat{\section}       % Command to format sections
  {\normalfont\bfseries\fontsize{9}{10}\selectfont}          % Font size and style for section titles
  {\thesection}              % Section number format
  {0.5em}                      % Spacing between number and title
  {}                         % Code before the title

% Tags for the mat expressions, use [] instead
\newtagform{brackets}{[}{]}
\usetagform{brackets}

\title

\begin{document}

\noindent
\textbf{Psychological Factors and COVID-19 Test Outcomes: Associations and Differences}\\
\vspace{0.1em} 
\noindent\hrulefill 
\vspace{0.8em} 

\begin{center}
    \textbf{Eurico Martins, Gutelvam Fernandes, Hugo Alves}\\
    \vspace{0.5em}
    Master in Applied Artificial Intelligence;\\
		EST - Instituto Politécnico do Cávado e do Ave, Portugal.\\
    \vspace{0.5em}
    \href{mailto:a8794@alunos.ipca.pt}{a8794@alunos.ipca.pt}, \href{mailto:a33791@alunos.ipca.pt}{a33791@alunos.ipca.pt}, \href{mailto:a30783@alunos.ipca.pt}{a30783@alunos.ipca.pt}
\end{center}

\vspace{3em}
\noindent {\bfseries \fontsize{9}{11}\selectfont ABSTRACT}
\vspace{1em}\newline
This study investigates the relationship between psychological factors—stress, anxiety, depression, and optimism—and COVID-19 test outcomes. Correlation analyses identified significant associations among the psychological variables, though their correlations with testing positive for COVID-19 were weak. Stress, anxiety, and depression exhibited minimal negative associations, while optimism demonstrated a positive correlation with positive test results. Normality tests revealed deviations from normality, and Mann-Whitney U tests indicated significant group differences in stress, depression, and optimism, while anxiety showed no statistically significant difference. These findings highlight nuanced links between psychological health and COVID-19 testing outcomes.

\vspace{1em}
\textbf{Keywords}: Psychological factors, stress, anxiety, depression, COVID-19

\vspace{2em}
\section{INTRODUCTION}
\vspace{0.5em}
The COVID-19 pandemic has had a profound impact on both physical and mental health worldwide.
While the physical health consequences are well-documented, the psychological effects have also been significant, with heightened levels of depression, anxiety, and stress reported across the globe.
Social isolation, economic uncertainty, and ongoing health fears have all contributed to a marked increase in mental health challenges, exacerbating existing vulnerabilities.
The pandemic disrupted daily routines, created widespread financial and social stress, and fostered an environment of uncertainty, leading to substantial deterioration in psychological well-being for many individuals.
\vspace{0.5em}\newline
Depression emerged as one of the most prevalent psychological effects of the pandemic.
Social isolation, coupled with uncertainty surrounding the virus, intensified feelings of helplessness and loneliness. Economic challenges, such as job losses and financial insecurity, further contributed to the surge in depressive symptoms.
The necessary social distancing measures, though essential for preventing virus spread,
left many without the support systems that might have helped mitigate feelings of isolation, leading to a global rise in depression during the pandemic.
\vspace{0.5em}\newline
Anxiety also became a major issue, driven by the constant flow of information about the virus and the fear of its unpredictable impact on health.
The concern about personal and familial well-being, along with economic instability, led to heightened anxiety levels.
Unlike transient anxiety, the persistent nature of the pandemic resulted in chronic distress, affecting individuals' ability to cope with daily life and function normally.
These prolonged periods of anxiety added to the overall mental health burden of the crisis.
\vspace{0.5em}\newline
Dispositional optimism, or the general expectation that good things will happen, emerged as a key protective factor against the mental health challenges of the pandemic.
Optimistic individuals tend to cope with adversity in more adaptive ways, maintaining a sense of hope and agency during difficult times.
Research has shown that optimism is associated with better psychological outcomes, including lower stress levels and greater resilience. During the pandemic,
optimism allowed individuals to manage uncertainty more effectively, promoting healthier behaviors and improving overall well-being.
\vspace{0.5em}\newline
Stress, an inevitable response to perceived challenges, has been another significant consequence of the pandemic.
The disruption of daily life, fear of illness, and uncertainty about the future combined to create a chronic state of stress for many. Prolonged stress has been linked to various physical and mental health issues,
including cardiovascular problems and weakened immune systems, further exacerbating the psychological toll of the pandemic.
The persistent nature of stress during this period has contributed to a rise in stress-related disorders globally.
\vspace{0.5em}\newline
The intersection of COVID-19 with mental health issues like depression, anxiety, and stress highlights the need for psychological resilience.
While the pandemic has undeniably worsened global mental health, optimism has proven to be a key factor in mitigating its effects. Further research into 
how optimism influences mental health could offer valuable strategies to strengthen resilience and better equip individuals to cope with the ongoing challenges posed by global crises.
\end{document}